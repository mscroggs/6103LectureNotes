\documentclass[11pt,a4paper,oneside]{book}
\usepackage{lecturestyle}


\begin{document}


This document is intended as a guide to what should be covered in each lecture.
It is based on there being 30 hours of lectures in total.
If this total changes, this guide will need adjusting.

\begin{tabular}{|p{.15\textwidth}|p{.8\textwidth}|}
  \hline
  \textbf{Lecture} & \textbf{Content} \\  \hline
  1 & Functions; polynomials of degree 0 and 1 \\  \hline
  2 & Polynomials of degree 2 \\  \hline
  3 & Polynomials of degree 3 and higher \\  \hline
  4 & Fractional and negative indices; graphs of exponential functions\\  \hline
  5 & Radians; $\sin$, $\cos$ and $\tan$; polar co-ordinates\\  \hline
  6 & Differentiation by first principles; rules for differentiating polynomials\\  \hline
  7 & Sum rule; product rule; chain rule\\  \hline
  8 & Combining the rules; applications of differentiation;\\  \hline
  9 & Differentiating inverse functions; differentiating parametric functions (including polar co-ordinates)\\  \hline
  10 & Real life examples; exponentials and logs\\ \hline
  11 & Differentiating $e^x$; $\ln x$; differentiating other exponentials and logs; exponential growth and decay\\  \hline
  12 & Integration as the area under a curve; integration as finding the antiderivative; integrating polynomials, $\sin$ and $\cos$\\  \hline
  13 & Integration by substitution\\  \hline
  14 & The $\ln$ special case; trigonometric substitutions\\  \hline
  15 & \textit{[catch up lecture; or further practice of substitution]}\\  \hline
\end{tabular}

\vspace{-5mm}
\begin{center}\textbf{Reading week}\end{center}
\vspace{-5mm}

\begin{tabular}{|p{.15\textwidth}|p{.8\textwidth}|}
  \hline
  16 & Integration by parts\\  \hline
  17 & Partial fractions\\  \hline
  18 & Which rule to use; difficulties of definite integration\\  \hline
  19 & Applications of integration\\  \hline
  20 & Numerical integration; the trapezium rule\\  \hline
  21 & Differential equations (ODEs); separation of variables\\  \hline
  22 & Integrating factors\\   \hline
  23 & Linear and non-linear ODEs; homogeneous and non-homogeneous equations; particular integrals and complementary functions\\  \hline
  24 & Finding complementary functions for second order linear ODEs with constant coefficients\\  \hline
  25 & Finding complementary functions when the auxiliary equation doesn't have two distinct real roots\\   \hline
  26 & Finding particular integrals for second order linear ODEs with constant coefficients\\  \hline
  27 & Numerical ODEs; Euler's method\\  \hline
  28 & Simple harmonic motion; other applications of ODEs\\  \hline
  29 & \textit{[catch up lecture; or further practice]}\\  \hline
  30 & \textit{[Looking back; what's next; my research (numerical PDEs); Christmas flexagons]}\\  \hline
\end{tabular}

\end{document}
