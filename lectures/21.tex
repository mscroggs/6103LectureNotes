\chapter{Differential Equations}

In many applications, we have equations relating a functions and its derivatives. For example:

\begin{itemize}
\item[ ] In radioactive decay, we know $\displaystyle \frac{dy}{dt} = \lambda y$, where $y$ is the number of particles of radioactive material.
\item[ ] Inflation is expressed as a percentage of current prices, so $\displaystyle \frac{dp}{dt} = i p$, where $p$ is prices and $i$ is inflation.
\item[ ] The movement of an object on a spring follows the equation $\displaystyle \frac{d^2 y}{dx^2} = -\omega y$.
\end{itemize}

Equations like these are called \textbf{(ordinary) differential equations} or \textbf{ODEs}.

In this chapter we will look at methods for solving ODEs.

\section{Terminology}
\begin{definition}
An equation involving $y$ and $\displaystyle\frac{dy}{dx}$ is called a \textbf{first order} ODE.

An equation involving $y$, $\displaystyle\frac{dy}{dx}$ and $\displaystyle\frac{d^2y}{dx^2}$ is called a \textbf{second order} ODE.

When solving ODEs, solutions involving constants often appear. These are called \textbf{general solutions} of ODEs.
\end{definition}

Extra information is often given to give the constants in the general solution a value.

\begin{definition}
The extra information given is called the \textbf{boundary conditions}.

A problem with an ODE and boundary conditions is called an \textbf{initial value problem} or \textbf{IVP}.
\end{definition}

\begin{definition}
An $n$-th order differential equation is linear if it can be written in the form:
\begin{equation*}
y^{(n)}+a_{n-1}(x)y^{(n-1)}+a_{n-2}(x)y^{(n-2)}+\dots+a_{1}(x)y'+a_{0}(x)y=f(x),
\end{equation*}
or
\begin{equation*}
\frac{d^ny}{dx^n}+a_{n-1}(x)\frac{d^{n-1}y}{dx^{n-1}}+a_{n-2}(x)\frac{d^{n-2}y}{dx^{n-2}}+\dots+a_1(x)\frac{dy}{dx}+a_0(x)y=f(x),
\end{equation*}
where $a_i$ ($i=0,1,2,\dots,n$) and $f$ are known functions of $x$. 
\end{definition}

\begin{example}
\(y'+2y=e^x\) is a first-order linear differential equation.

\(yy'=x\) is a first-order non-linear differential equation.

\(y'' - e^xy' + y=x\) is a second-order linear differential equation.
\end{example}

\begin{definition}
If $f(x)\equiv0$, then the differential equation is said to be \textbf{homogeneous}; otherwise, we say the equation is \textbf{non-homogenous} or \textbf{inhomogenous}.
\end{definition}

\begin{example}
\(y'+2y=e^x\) is a non-homogeneous differential equation.

\(y'+2y=0\) is a homogeneous differential equation.
\end{example}

\section{First order differential equations}
Here we will consider different techniques to solve first order ODEs.

\subsection{Separation of variables}
First order ODEs can be written in the form \[\frac{dy}{dx}=f(x,y).\]
For example
\begin{itemize}
\item[ ] $y'=-2xy+e^x$,
\item[ ] $\frac{dy}{dx}=\pm \sqrt{x^3-2\ln y+4e^x}$,
\item[ ] $y'=x/y^2$.
\end{itemize}

\begin{definition}
A function $f(x,y)$ is \textbf{separable} if it can be written as
\[f(x,y)=g(x)h(y).\]
\end{definition}

\begin{example}
Let \(\displaystyle f(x,y)=\frac{x}{y^2}=x\cdot\frac{1}{y^2}\).

This is seperable because
\[\frac{x}{y^2} = x\cdot\frac{1}{y^2}\]
so \(f(x,y) = g(x)h(y)\) where
\begin{align*}
g(x) &= x\\
h(y) &= \frac{1}{y^2}
\end{align*}
\end{example}

When $f$ is seperable, we can solve \(\displaystyle\frac{dy}{dx}=f(x,y)\) by a method called \textbf{separating the variables}.

\begin{example}
Consider the differential equation 
\[y'=\lambda y.\]
We already know the solution to this equation. Now let us see how to derive it using separation of variables.
\[\frac{dy}{dt}=\lambda y,\]
taking all things relating to $y$ to the left, and for $t$ to the right, we have
\[\frac{1}{y}dy=\lambda dt.\]
Integrating both sides we have
\[\int \frac{1}{y}\dx[y]=\int\lambda\dx[t],\]
hence, using what we have learnt in previous chapters we get
\[\ln y=\lambda t+C.\]
Finally, re-arranging for $y$, we have
\[y=e^{\lambda t+C}=Ae^{\lambda t},\quad A=e^C.\]
\end{example}

\begin{example}
Consider the equation
\[\frac{dy}{dx}=xy,\]
following the procedure as in the previous example, we have
\[\frac{1}{y}\dx[y]=x\dx.\]
Integrating both sides we have
\begin{eqnarray*}
&&\int\frac{1}{y}\dx[y]=\int x\dx,\\
&\implies& \quad \ln y=\frac{1}{2}x^2 +C.
\end{eqnarray*}
Taking exponentials of both sides in order to re-arrange for $y$, we get
\[y=e^{\frac{1}{2}x^2+C}=Ae^{\frac{1}{2}x^2},\quad A=e^C.\]
We can check if this satisfies the original equation:
\[\frac{dy}{dx}=\frac{d}{dx}\left(Ae^{\frac{1}{2}x^2}\right)=Ae^{\frac{1}{2}x^2+}\cdot\frac{1}{2}\cdot2x=xy.\]
\end{example}

\begin{example}
Consider the differential equation
\[y^2y'=x.\]
We first write it in the form $y'=f(x,y)$, i.e.
\[\frac{dy}{dx}=\frac{x}{y^2},\]
now we realises that we can apply separation of variable, so
\[y^2dy=x\dx,\]
\begin{eqnarray*}
&\implies& \quad \int y^2dy=\int x\dx\\
&\implies& \quad \frac{1}{3}y^3=\frac{1}{2}x^2+C,\\
&\implies& \quad y=\left(\frac{3}{2}x^2+C'\right)^{\frac{1}{3}},
\end{eqnarray*}
where $C'$ is some constant (different to $C$, since we multiplied through by $3$). Again, we check the solution satisfies the equation
\begin{eqnarray*}
\frac{dy}{dx}&=&\frac{d}{dx}\left[\left(\frac{3}{2}x^2+C'\right)^{\frac{1}{3}}\right] \\
&=&\frac{1}{3}\left(\frac{3}{2}x^2+C'\right)^{-\frac{2}{3}}\cdot\frac{3}{2}\cdot2x \\
&=& x\left(\frac{3}{2}x^2+C'\right)^{-\frac{2}{3}} \\
&=& x\left[\left(\frac{3}{2}x^2+C'\right)^{\frac{1}{3}}\right]^{-2} \\
&=&\frac{x}{y^2}.
\end{eqnarray*}
\end{example}

\begin{example}
Consider the following initial-value problem:
\[\frac{dy}{dx}=y^2(1+x^2),\quad y(0)=1.\]
First, we find the general solution, note, we can use separation of variables in this example, so
\[\frac{1}{y^2}\dx[y]=(1+x^2)\dx\]
\begin{eqnarray*}
&\implies& \quad \int\frac{1}{y^2}\dx[y]=\int(1+x^2)\dx \\
&\implies& -\frac{1}{y}=x+\frac{1}{3}x^3+C \\
&\implies&\quad y=-\frac{1}{x+\frac{1}{3}x^3+C}.
\end{eqnarray*}
Now check that the general solution satisfies the original differential equation:
\[\frac{dy}{dx}=\frac{d}{dx}\left(-\frac{1}{x+\frac{1}{3}x^3+C}\right)=\frac{1+x^2}{\left(x+\frac{1}{3}x^3+C\right)^2}=(1+x^2)y^2.\]

Now it remains to find the constant $C$, by applying the condition $y(0)=1$, i.e. we put $x=0$.
\[y(0)=-\frac{1}{0+\frac{1}{3}\cdot0^3+C}=-\frac{1}{C}=1,\quad\implies\quad C=-1.\]
So the solution to the initial value problem is
\[y=\frac{1}{1-x-\frac{1}{3}x^3}.\]
\end{example}

\begin{example}
Consider the initial value problem
\[e^y y'=3x^2,\quad y(0)=2.\]
First, find the general solution,
\begin{eqnarray*}
&&e^y y'=3x^2 \\
&\implies& \frac{dy}{dx}=3x^2e^{-y} \\
&\implies& e^y\dx[y]=3x^2\dx \\
&\implies& \int e^y\dx[y]=\int 3x^2\dx \\
&\implies& e^y=x^3+C\\
&\implies& y=\ln(x^3+C).
\end{eqnarray*}
Check:
\[\frac{dy}{dx}=\frac{d}{dx}\left( \ln(x^3+C)\right)=\frac{3x^2}{x^3+C}=3x^2\frac{1}{x^3+C}.\]
Recall $e^{\ln(a)}=a$, using this, we can write
\[\frac{dy}{dx}=3x^2e^{\ln\left( \frac{1}{x^3+C}\right)}=3x^2e^{-\ln(x^3+C)}=3x^2e^{-y}.\]
Now we apply the initial condition,
\[y(0)=\ln(C)=2\quad \implies\quad C=e^2,\]
so we have the final solution
\[y(x)=\ln (x^3+e^2).\]
\end{example}

