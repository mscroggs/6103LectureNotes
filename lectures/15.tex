\subsection{Trigonometric substitution}
\begin{example}
We know that 
\[\int \frac{1}{\sqrt{1-x^2}}dx=\sin^{-1}x+c,\]
since 
\[\frac{d}{dx}(\sin^{-1}x)=\frac{1}{\sqrt{1-x^2}}.\]

Actually, we can work out this integral by a substitution like $x=\sin u$ because we know that
\[1-\sin^2u=\cos^2u,\]
and 
\[\frac{dx}{du}=\cos u\quad \text{or} \quad dx=\cos u\dx[u].\]
Thus, we calculate the integral as
\[\int \frac{1}{\sqrt{1-x^2}}\dx=\int\frac{\cos u}{\sqrt{1-\sin^2u}}\dx[u]=\int\dx[u]=u+c=\sin^{-1}x+c.\]
\end{example}

\begin{example}
\[\int\frac{1}{1+x^2}\dx=\tan^{-1}x+c,\quad \text{since}\quad \frac{d}{dx}(\tan^{-1}x)=\frac{1}{1+x^2}.\]
Let us try the following
\[x=\tan\theta\]
\[\frac{dx}{d\theta}=\frac{1}{\cos^2\theta}=1+\tan^2\theta\]
\[dx=(1+\tan^2\theta)\dx[\theta]\]
\[\int\frac{1}{1+x^2}\dx=\int\frac{1}{1+\tan^2\theta}(1+\tan^2\theta)\dx[\theta]=\int\dx[\theta]=\theta+c=\tan^{-1}x+c.\]
\end{example}

\begin{example}
\[\int\frac{1}{1+2x^2}\dx\]
This is similar to the previous example. If we try
\[\sqrt{2}x=\tan\theta\]
\[dx=\frac{1}{\sqrt{2}}(1+\tan^2\theta)\dx[\theta]\]
\[\theta=\tan^{-1}(\sqrt{2}x)\]
\begin{eqnarray*}
\int\frac{1}{1+2x^2}\dx &=& \int\frac{1}{1+(\sqrt{2}x)^2}\dx \\
&=&\frac{1}{\sqrt{2}}\int\frac{1}{1+\tan^2\theta}(1+\tan^2\theta)\dx[\theta] \\
&=& \frac{1}{\sqrt{2}}\theta+c \\
&=& \frac{1}{\sqrt{2}}\tan^{-1}(\sqrt{2}x)+c.
\end{eqnarray*}
\end{example}


