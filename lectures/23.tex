\section{Complementary functions and particular integrals}

In the previous three examples, we gained the following results:
\[y=\frac{1}{3}x^2+c\cdot.\frac{1}{x},\]
\[y=1+c\cdot e^{-\frac{1}{2}x^2},\]
\[y=-(x^2+2x+2)+ce^x.\]

These examples have something very important in common, that is the solutions have the following form
\[y=f(x)+cg(x),\]
with explicit functions $f$ and $g$.

\begin{definition}
When \(y=f(x) + cg(x)\) is the solution of an ODE, \(f\) is called the \textbf{particular integral} (P.I.) and \(g\) is called the \textbf{complementary function} (C.F.).
\end{definition}

We can use particular integrals and complementary functions to help solve ODEs if we notice that:
\begin{enumerate}
\item The complementary function ($g$) is the solution of the homogenous ODE.
\item The particular integral ($f$) is any solution of the non-homogenous ODE.
\end{enumerate}

\begin{example}
We will use complementary functions and particular integrals to solve
\[y'+\lambda y=p(x),\quad \lambda \text{ is constant.}\]
We know that the general solution is 
\[y(x)=\underbrace{f(x)}_{\text{particular integral (P.I.)}}+\underbrace{Cg(x)}_{\text{complementary function (C.F.)}},\]
where
\[f'+\lambda f=p(x),\]\[g'+\lambda g=0,\]
and $C$ is the constant of integration.

We start by finding $g$. We need to solve
\[g'+\lambda g=0.\]
The solution of this is
\[g=Ce^{-\lambda x}.\]
[This can be found be separating the variables or by inspection.]

Therefore, the general solition to $y'+\lambda y=p(x)$ is
\[y=f(x)+Ce^{-\lambda x},\]
where $f$ is a particular integral.
\end{example}

The value of $f$ depends on $p$. To find $f$, we make a guess at what it might look like then see if we are right.

\begin{example}[$p(x)=x$]
Consider the differential equation
\[y'+y=x,\]
so we have
\[\lambda=1,\quad p(x)=x.\]
The solution of this ODE will be of the form \[y=f(x)+Ce^{-x}.\]

We can guess that $f$ should be a polynomial with a degree of one, since $p(x)=x$. So we try the most general first order polynomial, $f(x)=ax+b$, and so $f'(x)=a$. Substituting $y=f(x)$ into the differential equation we have that
\begin{align*}f'+f&=a+ax+b\\&=ax+(a+b).\end{align*}
And so \[x\equiv ax+(a+b).\]
Thus, comparing coefficients from the LHS and RHS we must have that
\begin{align*}
    \left. \begin{array}{l}
	a=1 \\
	a+b=0
    \end{array}\right\}
    \quad \implies \quad
        \left. \begin{array}{l}
	a=1 \\
	b=-1
    \end{array}\right\} 
    \quad\implies \quad f(x)=x-1,
\end{align*}
so $f(x)=x-1$ is the particular integral. Therefore, the general solution to the original equation is
\[y(x)=x-1+Ce^{-x}.\]
\end{example}

When we solve higher order linear ODEs, we use a similar method:]
\begin{enumerate}
\item We solve the homogenous equation to find the complementary function.
\item We guess the form of the particular integral then try it.
\end{enumerate}
