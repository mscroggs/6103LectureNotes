\subsection{Integrating factors}
Homogenous first order linear ODEs can be solved by seperating the variables. Non-homogenous ODEs cannot. In this section we will look at how to 
solve non-homogenous first order linear ODEs.

A first order linear ODE looks like
\begin{equation*}
\frac{dy}{dx}+g(x)y=f(x).
\end{equation*}

When \(f(x)\) is not zero, the ODE is non-homogenous.
First order non-homogenous ODEs can be solved using integrating factors.

\begin{definition}
The \textbf{integrating factor} of the ODE 
\[\frac{dy}{dx}+g(x)y=f(x).\]
is 
\[\exp\left({\int g(x)\dx}\right).\]
When finding the integrating factor we can ignore the constant of integration.
\end{definition}

If we multiply each term by the integrating factor, it should make our equation easier to solve.
To solve 
\[\frac{dy}{dx}+g(x)y=f(x),\]
let \[T(x)=\int g(x)\dx,\]
so that the integrating factor is \[e^{T(x)}.\]
Multiplying through by the integrating factor gives
\[e^{T(x)}\frac{dy}{dx}+e^{T(x)}g(x)y=e^{T(x)}f(x),\]

By the product rule, we can find the derivative of $e^{T(x)}y$:
\begin{align*}
\frac{d}{dx}\left(e^{T(x)}y\right) &=e^{T(x)}\frac{dy}{dx}+y\frac{d}{dx}\left(e^{T(x)}\right) \\
&=e^{T(x)}\frac{dy}{dx}+ye^{T(x)}\frac{d}{dx}\left(T(x)\right) \\
&=e^{T(x)}\frac{dy}{dx}+ye^{T(x)}q(x)
\end{align*}
This is exactly what we had on the left hand side of the ODE after multiplying by the integrating factor. Therefore:
\begin{align*}
\frac{d}{dx}\left(e^{T(x)}y\right) &= e^{T(x)}f(x)\\
e^{T(x)}y &= \int e^{T(x)}f(x)\dx\\
y &= \frac{\int e^{T(x)}f(x)\dx}{e^{T(x)}}\\
\end{align*}

If we know how to integrate \(e^{T(x)}f(x)\), then we can use this method to solve the ODE.

\begin{example}
Consider the differential equation
\[\frac{dy}{dx}+\frac{y}{x}=x.\]
[note that $f(x,y)=x-(y/x)$ can't be separated.]

The integrating factor is:
\begin{align*}
\exp\left(\int\frac{1}{x}\dx\right)
&=\exp\left(\ln x\right)\\
&=x
\end{align*}

Multiplying through by the integrating factor gives:
\[x\frac{dy}{dx}+y=x^2\]

Notice that: \[\frac{d}{dx}\left(xy\right) = x\frac{dy}{dx}+y\]

Therefore:
\begin{align*}
\frac{d}{dx}\left(xy\right) &= x^2\\
xy &= \int x^2\dx\\
&= \frac{x^3}{3} +c\\
y &= \frac{x^2}{3} +\frac{c}{x}\\
\end{align*}

\end{example}

\begin{example}
\begin{align*}
\frac{dy}{dx}+xy&=x\\
\exp\left(\int x\dx\right)\frac{dy}{dx}+\exp\left(\int x\dx\right)xy&=\exp\left(\int x\dx\right)x\\
\exp\left(\frac{1}{2}x^2\right)\frac{dy}{dx}+\exp\left(\frac{1}{2}x^2\right)xy&=\exp\left(\frac{1}{2}x^2\right)x\\
\frac{d}{dx}\left(\exp\left(\frac{1}{2}x^2\right)y\right)&=\exp\left(\frac{1}{2}x^2\right)x\\
\exp\left(\frac{1}{2}x^2\right)y&=\int \exp\left(\frac{1}{2}x^2\right)x\dx\\
\exp\left(\frac{1}{2}x^2\right)y&=\exp\left(\frac{1}{2}x^2\right)\\
y&=\frac{\exp\left(\frac{1}{2}x^2\right)+c}{\exp\left(\frac{1}{2}x^2\right)}\\
y&=1+c\exp\left(-\frac{1}{2}x^2\right)
\end{align*}
Or
\[y = 1+c e^{-\frac{1}{2}x^2}\]
\end{example}

\begin{example}
Solve the initial-value problem
\[y'=y+x^2,\quad y(0)=1.\]

First, find the general solution:
\begin{align*}
\frac{dy}{dx} - y &= x^2\\
\exp\left(\int -1 \dx\right)\frac{dy}{dx} - \exp\left(\int -1 \dx\right)y &= \exp\left(\int -1 \dx\right)x^2\\
e^{-x}\frac{dy}{dx} - e^{-x}y &= e^{-x}x^2\\
\frac{d}{dx}\left(e^{-x}y\right) &= e^{-x}x^2\\
e^{-x}y &= \int e^{-x}x^2\dx\\
&= -e^{-x}x^2 + \int2xe^{-x}\dx\\
&= -e^{-x}x^2 - 2xe^{-x} + \int2e^{-x}\dx\\
&= -e^{-x}x^2 - 2xe^{-x} -2e^{-x}+c\\
y &= -x^2 - 2x -2+ce^x
\end{align*}

Now we must use the initial condition to find \(c\):
\begin{align*}
y &= -x^2 - 2x -2+ce^x\\
1 &= -0^2 - 2\cdot0 -2+ce^0\\
1 &= -2+c\\
3 &= c
\end{align*}

Therefore the solution to the problem is \[y = -x^2 - 2x -2+3e^x.\]
\end{example}

