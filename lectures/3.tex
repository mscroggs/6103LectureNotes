\subsubsection{Degree $\ge 3$}

In general, we have the algebraic equation
\begin{equation}
a_0+a_1x+a_2x^2 + \dots + a_nx^n = 0,
\end{equation}
which has $n$ roots, including real and complex (imaginary numbers, $z=\alpha + i\beta$) roots.
\begin{itemize}
\item[] $n=2$ \quad we have formulae for roots \quad (quadratics)
\item[] $n=3$ \quad we have formulae for roots \quad (cubic)
\item[] $n=4$ \quad we have formulae for roots \quad (quartics)
\item[] $n>4$ \quad No general formulae exist  \,\,\quad (proven by \'{E}variste Galois)
\end{itemize}
But in any case, we may try factorisation to find the roots. We have the useful theorem:

\begin{theorem}{Factor theorem}
Let \(P\) be a polynomial of degree \(n\). For \(a\in\mathbb{R}\) (or \(C\)),\par
$$P(a) = 0\quad \text{if and only if}\quad P(x)=(x-a)Q(x)$$
where Q is a polynomial of degree \(n-1\).

\end{theorem}

Once one root is found, this theorem can be used to factorise the polynomial.
\begin{example} 
Consider $P(x)=x^3-8x^2+19x-12$. We know that $x=1$ is a solution to $P(x)=0$, then it can be shown that
\begin{equation*}
P(x)=(x-1)Q(x)=(x-1)(x^2-7x+12).
\end{equation*}
Here $P(x)$ is a cubic and thus $Q(x)$ is a quadratic.
\end{example}

The next examples show two methods of finding \(Q(x)\).
\begin{example}[Comparing coefficients]
Consider $P(x)=x^3-x^2-3x-1$. By observation, we know 
\[P(-1)=(-1)^3-(-1)^2-3(-1)-1=0.\]
So $x_1=-1$ is a root. Let us write
\[P(x)=(x+1)(ax^2+bx+c),\]
then multiplying the brackets we have
\[P(x)=ax^3+(a+b)x^2 + (b+c)x +c,\]
which should be equivalent to $x^3-x^2-3x-1$. Thus, comparing the corresponding coefficients we have
\begin{eqnarray*}
a &=& 1, \\
a+b &=& -1,  \\
b+c &=& -3,  \\
c &=& -1.
\end{eqnarray*}
This set of simultaneous equations has the solution
\[a=1,\quad b=-2,\quad c=-1.\]
So we can write
\[P(x)=(x+1)(x^2-2x-1)\]
To find the other two solutions of $P(x)=0$, we must set $(x^2-2x-1)=0$ which has solutions $x_{2,3}=1\pm\sqrt{2}$, together with $x_1=-1$ we have a complete set of solutions for $P(x)=0$.
\end{example}

As with many areas of mathematics, there are many ways to tackle a problem. Another way to find $q(x)$ given you know some factor of $P(x)$, is called {\it polynomial division}.
\begin{example}[Long division of polynomials]
Consider $P(x)=x^3-x^2-3x-1$, we know $P(-1)=0$. The idea is that we ``divide" $P(x)$ by the factor $(x+1)$, like so:\\[1em]
\begin{center}
\polylongdiv{x^3-x^2-3x-1}{x+1}
\end{center}
Hence, multiplying the {\it quotient} by the {\it divisor} we have $(x+1)(x^2-2x-1)=x^3-x^2-3x-1$. 
\end{example}

